\documentclass{article}
\usepackage{tikz}
\usepackage[en]{ukon-infie}
\usepackage[utf8]{inputenc}
\usepackage{algorithm2e}
\usepackage{amsmath}
\usepackage{graphicx}

\BottomPointtable 
\Names{Simon Giebenhain, } 		
\Lecture[EA]{Design and Analysis of Algorithms} 
\Term{WS 2017/18} 	   	
\Department{testdepartment} 
\Group{A}	


\begin{document}

    \begin{ukon-infie}[30/October/2017]{1}
    
\begin{exercise}[p=8]{Growth of Functions}	

 Uitilzing the formula from the lecture, one computes the following: \\
 
 With: $n = 8$, $\overline{x} = \frac{\sum_{i = 1}^8 i}{8} = 4.5$ and $\overline{y} = \frac{\sum_{i = 1}^8 y_i}{8} = 12.875$. This yields: \\
 $$ b = \frac{\sum_{i = 1}^n(y_i - \overline{y})(x_i - \overline{x})}{\sum_{i = 1}^n(x_i - \overline{x})^2} = 1.1071428$$, and
 $$ a = \overline{y} - b \overline{x} = 7.8928576$$
 
 We computed the values with the following java program: \\
 \begin{verbatim}
 float[] x = {1,2,3,4,5,6,7,8};
        float[] y = {10,11,11,10,13,14,16,18};
        float meanX = 4.5f;
        float meanY = 12.875f;

        float above = 0;
        for (int i = 0; i < x.length; i++) {
            above += (y[i] - meanY) * (x[i] - meanX);
        }
        float below = 0;
        for (int i = 0; i < x.length; i++) {
            below += (x[i] - meanX) * (x[i] - meanX);
        }

        float b = above / below;

        float a = meanY - b * meanX;

        System.out.println("a = " + a + ", b = " + b);
 \end{verbatim}
\end{exercise}


    \end{ukon-infie}
\end{document}
