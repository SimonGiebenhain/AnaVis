\documentclass{article}
\usepackage[de]{ukon-infie}
\usepackage[utf8]{inputenc}
\usepackage{algorithm2e}
\usepackage{amsmath}
\usepackage{graphicx}
% kann de oder en sein
% kann bubble break, topexercise sein

\Names{Jonas Probst, Simon Giebenhain}
\Lecture[AnaVis]{Analyse und Visualisierung von Informationen}
\Term{WS 2017/18}

\begin{document}
    \begin{ukon-infie}[29.11.17]{5}

        \begin{exercise}[p=10]{}
       
		\end{exercise}
		
		\begin{exercise}[p=10]{}
		a) und b)\\\\
		\begin{tabular}{|c|c|c|c|c|c|c|c|c|c|c|c|c|c|}
		\hline 
		 & 1 & 2 & 3 & 4 & 5 & 6 & 7 & 8 & 9 & 10 & Prediction k=1 & Prediction k=3 & Actual Class \\ 
		\hline 
		11 & $\sqrt[]{3}$ & $\sqrt[]{2}$ & $\sqrt[]{4}$ & $\sqrt[]{6}$ & $\sqrt[]{6}$ & $\sqrt[]{5}$ & $\sqrt[]{2}$ & $\sqrt[]{2}$ & $\sqrt[]{2}$ & $\sqrt[]{5}$ & N & N & P \\ 
		\hline 
		12 & $\sqrt[]{3}$ & $\sqrt[]{2}$ & $\sqrt[]{2}$ & $\sqrt[]{2}$ & $\sqrt[]{4}$ & $\sqrt[]{3}$ & $\sqrt[]{2}$ & $\sqrt[]{2}$ & $\sqrt[]{4}$ & $\sqrt[]{3}$ & N & P & P \\ 
		\hline 
		13 & $\sqrt[]{2}$ & $\sqrt[]{3}$ & $\sqrt[]{1}$ & $\sqrt[]{3}$ & $\sqrt[]{5}$ & $\sqrt[]{6}$ & $\sqrt[]{5}$ & $\sqrt[]{3}$ & $\sqrt[]{5}$ & $\sqrt[]{2}$ & P & P & P \\ 
		\hline 
		14 & $\sqrt[]{6}$ & $\sqrt[]{5}$ & $\sqrt[]{3}$ & $\sqrt[]{1}$ & $\sqrt[]{3}$ & $\sqrt[]{3}$ & $\sqrt[]{3}$ & $\sqrt[]{5}$ & $\sqrt[]{7}$ & $\sqrt[]{2}$ & P & P & N \\ 
		\hline 
		\end{tabular}\\\\
		c) \\
		k=1: $F_{TE}(K_1) = 3/4 =0.75$\\
		k=3: $F_{TE}(K_3) = 2/4 =0.5$\\
		
		\end{exercise}
		
		\begin{exercise}[p=4]{}
			
		\end{exercise}
		
		


		\begin{exercise}[p=3]{}
		

		\end{exercise}
		
		
\end{ukon-infie}
\end{document}
