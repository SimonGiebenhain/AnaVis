\documentclass{article}
\usepackage[de]{ukon-infie}
\usepackage[utf8]{inputenc}
\usepackage{algorithm2e}
\usepackage{amsmath}
\usepackage{graphicx}
% kann de oder en sein
% kann bubble break, topexercise sein

\Names{Jonas Probst, Simon Giebenhain}
\Lecture[AnaVis]{Analyse und Visualisierung von Informationen}
\Term{WS 2017/18}

\begin{document}
    \begin{ukon-infie}[08.11.17]{2}

        \begin{exercise}[p=9.5]{}
        	\question{}{Normalization brings data to a scale between 0 and 1, which makes comparing data easier.}
			\question{}{Normalization only makes sense for "Grösse" and "Geschlecht", all the other columns are either ordinal or nominal values. Normalizing makes comparing the data between columns easier, so it is useful in this case.}
			\question{}{$f_{lin}(x)=\frac{x-min}{max-min}$, $min=165$, $max=203$\\
			$f_{lin}(167)=0,0526$\\
			$f_{lin}(181)=0,4211$\\
			$f_{lin}(192)=0,7105$\\}
			\question{}{$f_{lin}(x)=\frac{x-min}{max-min}$, $min=52$, $max=72$\\
			$f_{lin}(57)=0,25$\\
			$f_{lin}(63)=0,55$\\
			$f_{lin}(68)=0,8$\\}
        \end{exercise}
        
        \begin{exercise}[p=3]{}
        	\textbf{Data cleaning:} Remove noise, deal with inconsitencies and missing values.\\
        	\textbf{Normalization:} Transform data to a standardized scale which is easier to work with.\\
        	\textbf{Data Reduction:} Reduce the number of data points or dimensions by sampling and removal of redundant columns.
		
		\end{exercise}
		
		\begin{exercise}[p=6]{}
		\question{}{The goal of sampling is to reduce the size of the data set while keeping the information content the same.}	
		\question{}{In probabilistic sampling every data point can randomly end up in the sample, while in non-probabilistic sampling some data points are choosen out of which the sample is drawn, so some data points have a porbability of zero to end up in the sample.}
		\question{}{\textbf{Systematic Random Sampling:} Advantage: Easy to implement; Disadvantage: Problem with periodicities\\
		\textbf{Cluster Random Sampling:} Advantage: Cheap method when it is geographically convenient; Disadvantage:  Least representative of the population\\
		\textbf{Random Sampling:} Should be used when there is only on column so no categories are possible and when there is no danger of missing a characteristic. Examples:\\ 1. List of heights of people.\\ 2. List of employee income without any other information given.\\
		\textbf{Stratified Random Sampling:}Should be used when there are mltiple columns, one of which puts the data in different categories. Examples:\\ 1. Income of people grouped by level of education\\ 2. Weight of people grouped by gender\\}

		\end{exercise}


		\begin{exercise}[p=2]{}

		\end{exercise}

		\begin{exercise}[p=4]{}
		
		\end{exercise}
		
		
\end{ukon-infie}
\end{document}
