\documentclass{article}
\usepackage[en]{ukon-infie}
\usepackage[utf8]{inputenc}
\usepackage{algorithm2e}
\usepackage{amsmath}
\usepackage{graphicx}
\usepackage{hyperref}
% kann de oder en sein
% kann bubble break, topexercise sein

\Names{Jonas Probst, Simon Giebenhain}
\Lecture[AnaVis]{Analyse und Visualisierung von Informationen}
\Term{WS 2017/18}

\begin{document}
    \begin{ukon-infie}[31.01.18]{11}

        \begin{exercise}[p=3]{Temporal Data}  
        \question{}
        {
        Whereas static temporal data is a view of historic observations and therefore does not change anymore, dynamic temporal data is data which is periodically updated.
      	}
      	
      	\question{}
      	{
      	Whereas linear temporal data assumes a starting point and tracks data until the present/future or some ending point, cyclic temporal data stores observations in recurrent time intervals/points.
      	}
      	\question{}
      	{
      	Time points have no duration (like a snapshot of that moment), where as time intervals are defined by a start and an ending point.
      	}
		\end{exercise}
		
		
		\begin{exercise}[p=7]{Choose a Visualization for Temporal Data}
		\question{}
		{
		The data is static, since it contains data of the past, that is, the data is not updated anymore.\\
		Futhermore it is linear, because the data starts from he 12th of August and ends at some unknown day, collecting data for every day inbetween.\\
		Open and close(value of first/last transaction per day) are time point data and High,Low and Volume are time interval data, because they take the whole day into account.
		}
		\question{}
		{
		}
		\end{exercise}
		
		\begin{exercise}[p=4]{Pixel-oriented Visualization}
		
		\end{exercise}
		
		
\end{ukon-infie}
\end{document}
