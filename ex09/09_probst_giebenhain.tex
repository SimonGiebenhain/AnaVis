\documentclass{article}
\usepackage[en]{ukon-infie}
\usepackage[utf8]{inputenc}
\usepackage{algorithm2e}
\usepackage{amsmath}
\usepackage{graphicx}
% kann de oder en sein
% kann bubble break, topexercise sein

\Names{Jonas Probst, Simon Giebenhain}
\Lecture[AnaVis]{Analyse und Visualisierung von Informationen}
\Term{WS 2017/18}

\begin{document}
    \begin{ukon-infie}[17.01.18]{9}

        \begin{exercise}[p=4]{Information Visualization Reference Model}  
       	Step 1: Transform Raw Data to useful Data Tables by standard filtering and data cleaning methods.\\\\
       	Step 2: Decide which data should be used in the later visualization and perform necessary transformations.\\\\
       	Step 3: Decide which visualization methods should be used and implement them using the data from previous steps.\\\\
       	Step 4: Evaluate you visualization and if necesarry adjust previous steps.
    	

		\end{exercise}
		
		\begin{exercise}[p=8]{Judging Visualizations}
		
		\end{exercise}
		
		\begin{exercise}[p=10]{Data Analysis and Visualization Task}
		
		\end{exercise}
		
		
\end{ukon-infie}
\end{document}
